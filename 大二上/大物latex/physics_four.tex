\documentclass{article}
\usepackage{amsmath}
\usepackage{amssymb}
\usepackage{graphicx}
\usepackage{ctex}
\usepackage{subfigure}
\begin{document}
\title{第四次作业}
\maketitle
\begin{minipage}[b]{0.5\linewidth}
学号:190320517\\
姓名:葛旭\\
班级:自动化五班\\
\end{minipage}
\hfill
\begin{minipage}[b]{0.5\linewidth}
\includegraphics[width=0.45\textwidth]{gexu.eps}\\
\end{minipage}
1.B\\
2.A\\
3.C\\
4.C\\
5.B\\
6.D\\
7.C\\
8.D\\
10.\[\begin{gathered}
d\frac{x}{{d'}} = (k - \frac{1}{2})\lambda  \hfill \\
\Rightarrow d\frac{{{x_1}}}{{d'}} = (5 - \frac{1}{2})\lambda  \hfill \\
\Rightarrow d\frac{{{x_2}}}{{d'}} = ( - 4 - \frac{1}{2})\lambda  \hfill \\
\Rightarrow d\frac{{{x_1} - {x_2}}}{{d'}} = 9\lambda  \hfill \\
\lambda  = d\frac{{{x_1} - {x_2}}}{{9d'}} = 633nm \hfill \\ 
\end{gathered} \]
颜色为红色\\
11.\[\begin{gathered}
d\frac{{{x_1} - {x_2}}}{{d'}} = 10\lambda  \hfill \\
\Rightarrow d = \frac{{10\lambda d'}}{{{x_1} - {x_2}}} = 1.34 \times {10^{ - 4}}m \hfill \\ 
\end{gathered} \]
13.\[\begin{gathered}
(1)\sin \theta  = n\sin {\theta _1} = \frac{1}{2} \hfill \\
\Rightarrow {\theta _1} = \arcsin \frac{1}{{2n}} = {23.99^ \circ } \hfill \\
(2)\nu  = \frac{c}{\lambda } = 5 \times {10^{14}}Hz \hfill \\
{v_n} = \frac{c}{n} = 2.44 \times {10^8}m/s \hfill \\
{\lambda _n} = \frac{\lambda }{n} = 488nm \hfill \\
(3)geometry{\text{  S = SA + BC + }}\frac{d}{{\cos \theta }} = 0.111m \hfill \\
\Delta  = SA + BC + AB \times n = 0.114m \hfill \\ 
\end{gathered} \]
16.\[\begin{gathered}
2{n_2}d = (k + \frac{1}{2})\lambda  \hfill \\
k = 0,{d_{\min }} = \frac{\lambda }{{4{n_2}}} = 99.6nm \hfill \\ 
\end{gathered} \]
17.\[\begin{gathered}
\frac{d}{L} = \frac{\lambda }{{2n\Delta x}},\Delta x = \frac{{4.295 \times {{10}^3}}}{{29}}m \hfill \\
\Rightarrow d = \frac{{\lambda L}}{{2\Delta x}} = 5.75 \times {10^{ - 5}}m \hfill \\ 
\end{gathered} \]
20.\[\begin{gathered}
\frac{\lambda }{{2b}} = \frac{\lambda }{{2nb'}} \hfill \\
b - b' = 0.5mm \hfill \\
(n - 1)b' = 0.5mm \hfill \\
b' = \frac{{0.5}}{{n - 1}} = 1.25mm \hfill \\
\theta  = \arctan (\frac{\lambda }{{2nb'}}) = 1.71 \times {10^{ - 4}}rad \hfill \\ 
\end{gathered} \]
22.\[\begin{gathered}
r = \sqrt {2dR}  = \sqrt {(\Delta  - \frac{\lambda }{2})R} ,\Delta  = (k + \frac{1}{2})\lambda  \hfill \\
r = \sqrt {kR\lambda }  \hfill \\
{r_4} - {r_1} = \sqrt {R\lambda }  = \Delta r \hfill \\
{r_4}' - {r_1}' = \sqrt {R\lambda '}  = \Delta r' \hfill \\
\frac{{\sqrt \lambda  }}{{\sqrt {\lambda '} }} = \frac{{\Delta r}}{{\Delta r'}} \Rightarrow \lambda ' = 546nm \hfill \\ 
\end{gathered} \]
23.\[\begin{gathered}
r = \sqrt {2dR}  \hfill \\
{\text{when it's filled with air,then }}\Delta {\text{ = 2d + }}\frac{\lambda }{2} \hfill \\
r = \sqrt {(\Delta  - \frac{\lambda }{2})R} ,\Delta  = k\lambda ,k = 10,r = \sqrt {\frac{{19}}{2}\lambda R}  \hfill \\
{\text{when it's filled with some fluid,then }}\Delta {\text{ = 2nd + }}\frac{\lambda }{2} \hfill \\
r = \sqrt {(\Delta  - \frac{\lambda }{2})R/n} ,\Delta  = k\lambda ,k = 10,r = \sqrt {\frac{{19}}{2}\lambda R/n}  \hfill \\
\frac{{{r_1}}}{{{r_2}}} = \sqrt n  = \frac{{1.4}}{{1.27}},n = 1.22 \hfill \\ 
\end{gathered} \]
24.\[\begin{gathered}
(1)2{n_2}d = \left\{ \begin{gathered}
k\lambda ,light,k = 0,1,2 \cdots  \hfill \\
(k + \frac{1}{2})\lambda ,dark,k = 0,1,2 \cdots  \hfill \\ 
\end{gathered}  \right. \hfill \\
d = 0{\text{,it's the light wave}} \hfill \\
{\text{(2)}}2{n_2}d = (k + \frac{1}{2})\lambda  \hfill \\
d = \frac{{(k + \frac{1}{2})\lambda }}{{2{n_2}}} \leqslant {d_m},k \leqslant \frac{{2{n_2}{d_m}}}{\lambda } - \frac{1}{2} = 3.9 \hfill \\
k = 0,1,2,3,{\text{so we can only see four dark circle}}{\text{.}} \hfill \\ 
\end{gathered} \]
26.\[\begin{gathered}
1100 \times \frac{\lambda }{2} = 0.31mm \hfill \\
\Rightarrow \lambda  = 563.6nm \hfill \\ 
\end{gathered} \]
27.\[\begin{gathered}
(1)b\sin \theta  = (2k + 1)\frac{\lambda }{2},\sin \theta  = \frac{x}{f} \hfill \\
b\frac{x}{f} = (2k + 1)\frac{\lambda }{2} \hfill \\
{\lambda _{\min }} = 400nm,k = 4.75 \hfill \\
{\lambda _{\max }} = 760nm,k = 2.27 \hfill \\
{\text{so k = 3 or 4}} \hfill \\
{\text{from }}b\frac{x}{f} = (2k + 1)\frac{\lambda }{2} \hfill \\
k = 3,\lambda  = 600nm \hfill \\
k = 4,\lambda  = 466.7nm \hfill \\
{\text{(2) }}k = 3,\lambda  = 600nm \hfill \\
k = 4,\lambda  = 466.7nm \hfill \\
{\text{(3) }}k = 3,\lambda  = 600nm,N = 2k + 1 = 7 \hfill \\
k = 4,\lambda  = 466.7nm,N = 2k + 1 = 9 \hfill \\ 
\end{gathered} \]
29.\[\begin{gathered}
b\sin \theta  = (2k + 1)\frac{\lambda }{2} \hfill \\
so{\kern 1pt} {\kern 1pt} b\sin \theta  = \frac{{7\lambda }}{2} = \frac{{5\lambda '}}{2} \hfill \\
\lambda  = \frac{5}{7}\lambda ' = 428.6nm \hfill \\ 
\end{gathered} \]
30.\[\begin{gathered}
(1)b\sin \theta  = (2k + 1)\frac{\lambda }{2},\sin \theta  = \frac{x}{f} \hfill \\
b\frac{x}{f} = (2k + 1)\frac{\lambda }{2} \Rightarrow x = (2k + 1)\frac{{f\lambda }}{{2b}} \hfill \\
k = 1,{x_1} = \frac{{3f{\lambda _1}}}{{2b}} = 3 \times {10^{ - 3}}m \hfill \\
k = 1,{x_2} = \frac{{3f{\lambda _2}}}{{2b}} = 5.7 \times {10^{ - 3}}m \hfill \\
\Delta x = 2.7 \times {10^{ - 3}}m \hfill \\
(2)d = \frac{{1cm}}{{1000}} = {10^{ - 5}}m \hfill \\
d\frac{x}{f} = k\lambda  \hfill \\
k = 1,{x_1} = \frac{{{\lambda _1}f}}{d} = 2 \times {10^{ - 2}}m \hfill \\
k = 1,{x_2} = \frac{{{\lambda _2}f}}{d} = 3.8 \times {10^{ - 2}}m \hfill \\
\Delta x' = 1.8 \times {10^{ - 2}}m \hfill \\ 
\end{gathered} \]
31.\[\begin{gathered}
{\theta _0} = 1.22\frac{\lambda }{D} = \frac{l}{d} \hfill \\
d = 4918m \hfill \\ 
\end{gathered} \]
34.\[\begin{gathered}
(1)b = \frac{{1 \times {{10}^{ - 3}}}}{{500}} = 2 \times {10^{ - 6}}m \hfill \\
d\sin \theta  = k\lambda  \hfill \\
\sin \theta  = \frac{{k\lambda }}{d} < 1,k < \frac{d}{\lambda } = 3.4 \hfill \\
so{\kern 1pt} {\kern 1pt} k = 3 \hfill \\
(2)d(\sin i \pm \sin \theta ) = k\lambda  \hfill \\
\Rightarrow |\sin \theta | < 1 \hfill \\
\Rightarrow k = 1or5 \hfill \\
(3)d\sin {\theta _1} = k{\lambda _{\min }},{\lambda _{\min }} = 400nm \hfill \\
d\sin {\theta _2} = k{\lambda _{\max }},{\lambda _{\max }} = 760nm \hfill \\
k = 1,{\theta _1} = \arcsin (\frac{{{\lambda _{\min }}}}{d}),{\theta _2} = \arcsin (\frac{{{\lambda _{\max }}}}{d}) \hfill \\
\Delta x = f(\tan {\theta _2} - \tan {\theta _1}) = 0.21 \hfill \\ 
\end{gathered} \]
35.\[\begin{gathered}
(1)d\sin \phi  = 2\lambda  \hfill \\
d = 6 \times {10^{ - 6}}m \hfill \\
(2)\frac{d}{b} = \frac{k}{{k'}} = \frac{4}{{k'}} = c \hfill \\
k' = 1,c = 4,{\text{then lost }} \pm 4{\text{,}} \pm 8{\text{,}} \pm 1{\text{2}} \cdots  \hfill \\
k' = 2,c = 2,{\text{then lost }} \pm 2{\text{,}} \pm 4{\text{,}} \cdots ({\text{not we want}}) \hfill \\
k' = 3,c = \frac{4}{3},{\text{then lost }} \pm 4{\text{,}} \pm 8{\text{,}} \pm 1{\text{2}} \cdots  \hfill \\
k' = 4,c = 1,{\text{then lost }} \pm 1{\text{,}} \pm 2{\text{,}} \pm 3 \cdots ({\text{not we want}}) \hfill \\
(3)d\sin \phi  = k\lambda ,|\sin \phi | = \frac{{k\lambda }}{d} < 1,|k| < |\frac{d}{\lambda }| = 10 \hfill \\
k = 0, \pm 1{\text{,}} \pm 2{\text{,}} \pm 3, \pm 5{\text{,}} \pm 6 \pm 7{\text{,}} \pm 9,{\text{add up to 15}} \hfill \\ 
\end{gathered} \]
38.\[\begin{gathered}
{I_0} \times \frac{1}{2} \times {(\cos {60^ \circ })^2} = {I_1} \hfill \\
{I_0} = 8{I_1} \hfill \\
{I_0} \times \frac{1}{2} \times {(\cos {30^ \circ })^2} \times {(\cos {30^ \circ })^2} = {I_0} \times \frac{1}{2} \times \frac{3}{4} \times \frac{3}{4} = \frac{9}{4}{I_1} \hfill \\
so{\kern 1pt} {\kern 1pt} {\kern 1pt} {I_0} = \frac{9}{4}{I_1} \hfill \\ 
\end{gathered} \]
39.设总光强为$I_0$,自然光光强为$I_1$,偏振光光强为$I_2$\\
\[\begin{gathered}
{I_0} = {I_1} + {I_2} \hfill \\
I = \frac{1}{2}{I_1} + {I_2}{\cos ^2}\alpha  \hfill \\
{I_{\max }} = \frac{1}{2}{I_1} + {I_2} \hfill \\
{I_{\min }} = \frac{1}{2}{I_1} \hfill \\
{I_{\max }} = 5{I_{\min }} \hfill \\
\Rightarrow {I_2} = 2{I_1} \hfill \\ 
\end{gathered} \]
所以自然光占$\frac{1}{3}$,偏振光占$\frac{2}{3}$
\end{document}